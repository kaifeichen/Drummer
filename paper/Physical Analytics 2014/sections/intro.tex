\section{Introduction}
\label{sec:intro}

Audio probing has been used for tens of years in various ways to help indirectly
inspect environments or objects. Sonar is used by vessels to leverage sound propagation 
time to detect underwater objects. Ultrasonography uses ultrasound to generate
diagnostic imaging of subcutaneous body structures in clinics. \fxnote{more use cases}


However, as an approach that has been proved to be very useful to interact with 
physical environment, the acoustic sensors in smartphones gain little attention in 
industry and 
have not been evolved as much as other sensors in terms of providing physical information.
Most acoustic applications is focusing on semantic information in voice, e.g. natural 
language processing and text-to-speech synthesis. These use cases require only audible
sounds with adequate 
accuracy, and have good baseline proof of feasibility that human brain can process it. 
\fxnote{cite papers in http://www.cs.virginia.edu/~smn8z/paper/auditeur-mobisys2013.pdf}
Whereas smart phones haven't made the step to evolve from a phone to a smart 
mobile device that could converts physical information into cyber world. 
Most phones do audible sound specific optimizations in operating systems or 
hardware firmwares, and provide no access to the raw audio data. Specifically, most phone
do audio noise cancellation, which makes it impossible to record the sound sent by itself.


Some recent work has been done on gesture detection based on their influence on sound 
\fxnote{cite UW and CMU Doppler}. 

Moreover, 
as an ubiquitous sensor, acoustic signals also have been investigated a lot as 
an input to infer physical world. As examples, Doppler effect is ideal input to 
gestures inference \cite{gupta2012soundwave, sun2013spartacus}.


With initial specific purpose for audible sound playing and recording, microphone 
and speaker on commercial phones has limitation on sampling frequency and accuracy. 
Unlike emerging and rapidly evolving heterogeneous sensors and processors in 
smart phones \cite{invensense, applem7}, acoustic sensors suffer from lacking of 
applications, which in turn impair the interest from industry to improve and reduce
cost of better quality acoustic sensors.





In this paper, we introduce several environment features we explored and discuss how 
they can provide great information to different mobile applications. 


% To learn the environment, computer vision techniques has been proposed to capture 
% and reconstruct things from video and images, such as objects 
% \cite{rothganger20063d}, scenes \cite{snavely2006photo}, and buildings 
% \cite{furukawa2009reconstructing}. These techniques require fairly amount of 
% data and heavy post-processing computation. In addition, depth information from 
% RGB-D cameras are used to enable interactive real time 3D visual modeling, 
% especially while RGB-D sensors are getting cheaper, such as Microsoft Kinect 
% \cite{du2011interactive, izadi2011kinectfusion}.


% However, RGB-D cameras are still not available for most people. And privacy 
% concerns could be raised when reconstructing visual model of open spaces, 
% especially in indoor environment, where private information can be more 
% heterogeneous compared to outdoor situation. More importantly, not all 
% applications actually require accurate and complete model of environment. 
% For example, with the knowledge of floor plan, we can infer what a person 
% is doing simply by acquiring a coarse estimation of her/his location. 
%Standing in front of projection screen implies she/he is presenting, 


We propose to use acoustic echo to probe the environment, and build coarse 3D model 
of the environment. Microphones and speakers are available on almost all phones, 
which makes crowd-sourcing possible. And 
sensing the echo of a sound pulse will bring up almost no privacy concerns. We are 
trying to look at how accurate we can achieve to reconstruct 3D in-room model, and 
how much benefits we can introduce to existing techniques.


3D modeling/inroom localization

We only focus on closed environment

Video: Privacy, Energy Consumption, Computation, Availability
