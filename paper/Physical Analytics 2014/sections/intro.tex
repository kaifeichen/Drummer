\section{Introduction}
\label{sec:intro}

\fxnote{if not many applications can be found. 
change to: calibration information is important for indoor localization, people are 
using different ways to do that, sound is one way that is better .}



% acoustic signal was used a lot for environment detection
Audio probing has been used for tens of years in various ways to help indirectly
inspect environments or objects. Sonar is used by vessels to leverage sound propagation 
time to detect underwater objects. Ultrasonography uses ultrasound to generate
diagnostic imaging of subcutaneous body structures in clinics. \fxnote{more use cases}
Given the physical attributes of audio signal, it is idea for use in environment detection.
Sound wave is longitudinal mechanical wave, so \fxnote{what?}. 


% acoustic signal should be used for environment detection but not
With ubiquitous audio devices (e.g.\ microphone and speaker) built in all phones, 
acoustic signal should also be used for 
environment detection as an service. Applications including indoor localization, occupancy detection,
and 3D model reconstruction, can benefit from the feedback of this service.
However, acoustic devices in mobile phones gain little attention in 
industry and have not been evolved as much as other components \cite{invensense, applem7} 
in terms of providing physical information.
Most acoustic applications is focusing on semantic information in voice, such as natural 
language processing and text-to-speech synthesis. These areas require only audible
sounds with adequate 
accuracy, and have good baseline proof of feasibility that human brain can process it
\cite{nirjon2013auditeur}.
Whereas smart phones haven't made the step to evolve from a phone to a smart 
mobile device that could converts physical information into cyber world using acoustic signal. 
Worse still, most phones do audible sound specific optimizations in operating systems or 
hardware firmware, and provide no access to the raw audio data. Specifically, mobile phone
tend to audio noise cancellation for communication and entertainment quality, which makes 
it impossible to record the sound sent by itself in application layer.


% current work on audio probing 
In spite of these impediments, several recent work has been done on inferring physical world 
information using acoustic signal feature. Examples include: gesture detection using Doppler effect
\cite{gupta2012soundwave, sun2013spartacus}, room shape reconstruction using echo \cite{dokmanic2013acoustic}, 
blah
 \fxnote{details about those work}


% our work
In this paper, we introduce {\em Drummer} - a mobile could service to enable several environment awareness 
features using acoustic signal. \fxnote{goal, arch, algorithm}



% summary
In summary, our contributions are:

\begin{itemize}

\item Propose several environment detection features and their algorithms.

\item Investigate how good we can get for each feature, and advocate industry to enable 
more raw information from hardware layer, and expand the functionality of audio devices, 
such as maximum sending and sampling frequency.

\end{itemize}


% roadmap


% To learn the environment, computer vision techniques has been proposed to capture 
% and reconstruct things from video and images, such as objects 
% \cite{rothganger20063d}, scenes \cite{snavely2006photo}, and buildings 
% \cite{furukawa2009reconstructing}. These techniques require fairly amount of 
% data and heavy post-processing computation. In addition, depth information from 
% RGB-D cameras are used to enable interactive real time 3D visual modeling, 
% especially while RGB-D sensors are getting cheaper, such as Microsoft Kinect 
% \cite{du2011interactive, izadi2011kinectfusion}.


% However, RGB-D cameras are still not available for most people. And privacy 
% concerns could be raised when reconstructing visual model of open spaces, 
% especially in indoor environment, where private information can be more 
% heterogeneous compared to outdoor situation. More importantly, not all 
% applications actually require accurate and complete model of environment. 
% For example, with the knowledge of floor plan, we can infer what a person 
% is doing simply by acquiring a coarse estimation of her/his location. 
%Standing in front of projection screen implies she/he is presenting, 


% We propose to use acoustic echo to probe the environment, and build coarse 3D model 
% of the environment. Microphones and speakers are available on almost all phones, 
% which makes crowd-sourcing possible. And 
% sensing the echo of a sound pulse will bring up almost no privacy concerns. We are 
% trying to look at how accurate we can achieve to reconstruct 3D in-room model, and 
% how much benefits we can introduce to existing techniques.


% 3D modeling/inroom localization

% We only focus on closed environment

% Video: Privacy, Energy Consumption, Computation, Availability
