\section{Introduction}
\label{sec:intro}


% different indoor localization is used
Driven by great potential of applications, indoor localization has been studied in a wide scope of aspects.
Trilateration combined with ranging technology, such as infrared and ultrasound,
yields fine grained geometric localization results when no obstacle is presented \fxnote{[infrared, ultrasound]}. 
Fingerprinting localizes by comparing a signature with pre-populated database of fingerprints, such as 
wireless signal strengths, special motion patterns, acoustic patterns, and pre-installed beacons \fxnote{[RADAR, spinloc, livesynergy]}. 
Dead reckoning uses inertial measurement units (IMU), e.g. accelerometer, compass, and gyroscope, to estimate 
velocity and orientation for trajectory estimation \fxnote{[Zee, Warkie markie]}. Targeted applications also vary. Accurate geometric 
localization is used for indoor robot manipulation or automation. Symbolic localization targets applications that requires 
location semantics, such as occupancy-driven HVAC system, energy foot print, activity recognition, etc. 



All these source of information are not going to work independently, and tend to be fused together with 
appropriate statistic models to generate more reliable localization results. 
Particle Filter and Kalman Filter can be used to calculate posterior distribution of location given 
combined observations from ranging, fingerprinting, and dead-reckoning. Similar to outdoor location-based services, 
a database of location semantic 
and geometric location \fxnote{[some service discovery database project]} can be used to convert results in between.
As an example, if activity recognition module figures out the user is using coffee machine, then the possibility 
that she is around geometric points near coffee machine will be higher. 




% we need semantic location
%However, the focuses of these work are not perfectly in line with requirements from location-based services (LBS). 
%Most of them are purely focusing on how to localize a geometric location, whereas 
%many indoor LBS also requires the semantic of the location rather than the location itself, such as room number,
%or a handicap door. \fxnote{say geometric is important, interaction, proximity social network} 
%One example of indoor LBS is occupancy driven HVAC (heating, ventilation, and air conditioning).
%Because HVAC control are at most room-level granularity, so estimating the number of room the 
%user is in will be more nature and helpful. For LBSs that requires geometric locations, semantic will also help
%in many cases. For instance, proximity based social network \fxnote{[cite]} need to ensure two people staying near
%to each other have no wall in between. 



% \fxnote{geometric location needs calibration}
% Geometric location can be done by: trilateration or triangulation, dense site survey + fingerprinting, dead reckoning.
% First one has problems when obstacles are presented, second suffers from training overhead. 3rd works, but it needs calibration. 
% step count, particle filter, kalman filter.


% Location semantics can be inferred in different approaches. One way is to get geometric location with 
% adequate accuracy. By using predefined location to semantic information database \fxnote{[cite]}, we can get semantic information
% by querying geometric locations, which is also adopted by the majority of outdoor location-based services.
% However, given much richer information and faster environment change, creating and maintaining such as a
% semantic database is not feasible in most buildings. On the other hand, many semantic information requires
% find grained geometric estimation that can hardly be archived by state-of-art localization techniques. For 
% example, energy foot-printing \fxnote{[cite]} is the notion of monitoring personal energy consumption
% according to her location and 
% the electronic devices (e.g. coffee machine, refrigerator, microwave) being used at that location. The 
% distance between these devices can be sub-meter, and no current geometric localization technique can achieve this 
% accuracy in general cases. Another idea to infer semantic location is from the physical features, and it is very application
% specific. For instance, a room is usually enclosed by walls, so you should be able to encode room number in 
% special light bulbs, and give perfect room-level localization. However, some office buildings use glasses to separate
% rooms. %Also ultrasound beacon may also work, but some buildings has no door 
% For energy foot-printing, one can actually either see (by camera) or hear (by microphone) what the user is doing.



% intro of Drummer
In this paper, we investigate how sounds provide more information about the location
to feed data fusion models. We introduce {\em Drummer}, a environment probing and listening service 
to assist both geometric calibration and semantic indoor localization inference. The basic idea is sound 
can be used to measure distance from walls, and also learn human activities that will create featured sound.
This solve all problems, because: it can 
provide continuous calibration, it can detect environment change, 
it provide room level fingerprinting, sound has plenty potential of inferring activity (fridge/coffee machine).


In summary, our contributions are: (i) it can provide continuous calibration. (ii) it can detect environment change. 
(iii) it provide room level fingerprinting. (iv) sound has plenty potential of inferring activity (fridge/coffee machine).
(v) provide huge potential for extension and improvements

Roadmap: Section 2 related work and their faults, Section 3 overview, Section 4 calibrations, Section 5 Physical inference, 
Section 6 evaluation. Section 7 Discussion. Section 8 Conclusion

% % acoustic signal was used a lot for environment detection
% Audio probing has been used for tens of years in various ways to help indirectly
% inspect environments or objects. Sonar is used by vessels to leverage sound propagation 
% time to detect underwater objects. Ultrasonography uses ultrasound to generate
% diagnostic imaging of subcutaneous body structures in clinics. \fxnote{more use cases}
% Given the physical attributes of audio signal, it is idea for use in environment detection.
% Sound wave is longitudinal mechanical wave, so \fxnote{what?}. 


% % acoustic signal should be used for environment detection but not
% With ubiquitous audio devices (e.g.\ microphone and speaker) built in all phones, 
% acoustic signal should also be used for 
% environment detection as an service. Applications including indoor localization, occupancy detection,
% and 3D model reconstruction, can benefit from the feedback of this service.
% However, acoustic devices in mobile phones gain little attention in 
% industry and have not been evolved as much as other components \cite{invensense, applem7} 
% in terms of providing physical information.
% Most acoustic applications is focusing on semantic information in voice, such as natural 
% language processing and text-to-speech synthesis. These areas require only audible
% sounds with adequate 
% accuracy, and have good baseline proof of feasibility that human brain can process it
% \cite{nirjon2013auditeur}.
% Whereas smart phones haven't made the step to evolve from a phone to a smart 
% mobile device that could converts physical information into cyber world using acoustic signal. 
% Worse still, most phones do audible sound specific optimizations in operating systems or 
% hardware firmware, and provide no access to the raw audio data. Specifically, mobile phone
% tend to audio noise cancellation for communication and entertainment quality, which makes 
% it impossible to record the sound sent by itself in application layer.


% % current work on audio probing 
% In spite of these impediments, several recent work has been done on inferring physical world 
% information using acoustic signal feature. Examples include: gesture detection using Doppler effect
% \cite{gupta2012soundwave, sun2013spartacus}, room shape reconstruction using echo \cite{dokmanic2013acoustic}, 
% blah
%  \fxnote{details about those work}


% % our work
% In this paper, we introduce {\em Drummer} - a mobile could service to enable several environment awareness 
% features using acoustic signal. \fxnote{goal, arch, algorithm}



% % summary
% In summary, our contributions are:

% \begin{itemize}

% \item Propose several environment detection features and their algorithms.

% \item Investigate how good we can get for each feature, and advocate industry to enable 
% more raw information from hardware layer, and expand the functionality of audio devices, 
% such as maximum sending and sampling frequency.

% \end{itemize}


% % roadmap


% To learn the environment, computer vision techniques has been proposed to capture 
% and reconstruct things from video and images, such as objects 
% \cite{rothganger20063d}, scenes \cite{snavely2006photo}, and buildings 
% \cite{furukawa2009reconstructing}. These techniques require fairly amount of 
% data and heavy post-processing computation. In addition, depth information from 
% RGB-D cameras are used to enable interactive real time 3D visual modeling, 
% especially while RGB-D sensors are getting cheaper, such as Microsoft Kinect 
% \cite{du2011interactive, izadi2011kinectfusion}.


% However, RGB-D cameras are still not available for most people. And privacy 
% concerns could be raised when reconstructing visual model of open spaces, 
% especially in indoor environment, where private information can be more 
% heterogeneous compared to outdoor situation. More importantly, not all 
% applications actually require accurate and complete model of environment. 
% For example, with the knowledge of floor plan, we can infer what a person 
% is doing simply by acquiring a coarse estimation of her/his location. 
%Standing in front of projection screen implies she/he is presenting, 


% We propose to use acoustic echo to probe the environment, and build coarse 3D model 
% of the environment. Microphones and speakers are available on almost all phones, 
% which makes crowd-sourcing possible. And 
% sensing the echo of a sound pulse will bring up almost no privacy concerns. We are 
% trying to look at how accurate we can achieve to reconstruct 3D in-room model, and 
% how much benefits we can introduce to existing techniques.


% 3D modeling/inroom localization

% We only focus on closed environment

% Video: Privacy, Energy Consumption, Computation, Availability
